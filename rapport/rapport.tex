\documentclass[a4paper,11pt]{article}

\usepackage[francais]{babel}
\usepackage[utf8]{inputenc}
\usepackage[pdftex]{graphicx}
\usepackage{hyperref}
\usepackage{color}
\usepackage{fancyheadings}
\usepackage{lastpage}
\usepackage{fullpage}

\newcommand{\HRule}{\rule{\linewidth}{0.5mm}}
\newcommand{\reporttitle}{Titre}
\newcommand{\reportauthor}{Auteurs}
\newcommand{\reportdate}{\today}

\lhead{}
\rhead{\leftmark \rightmark}
\cfoot{Page \thepage/\pageref{LastPage}}


\begin{document}	

\begin{center}

\begin{minipage}[t]{0.4\textwidth}
  \begin{flushleft} \large
    \reportauthor
  \vfill
  \end{flushleft}
\end{minipage}
\begin{minipage}[t]{0.5\textwidth}
  \begin{flushright}
  \includegraphics [width=30mm]{figures/tpt.jpg} \\[0.5cm]
  \end{flushright}
\end{minipage}
\HRule \\[0.5cm]
{\huge \bfseries \reporttitle}\\[0.3cm]
\HRule \\[1.5cm]

\end{center}

\section*{Introduction}

\section{Fonctionnement du cache}

"2.1 Data Caches"~\cite{brumley2009cache}\\
"2.1. Memory and Cache Structure"~\cite{tromer2010efficient}\\
Article Canteaut~\cite{canteaut2006understanding}

\section{Exemple d'attaque: le cas d'AES}

\begin{description}
\item[on récupère] la clé AES
\item[pourquoi] difficulté d'implémenter algo en temps constant (détailler les raisons) donc les traces du chiffrement (cache hit/miss) permettent d'en déduire des infos
\item[comment] détailler  ~\cite{canteaut2006understanding} + ~\cite{bernstein2005cache} + (\cite{tromer2010efficient} et~\cite{osvik2006cache}(mêmes auteurs))
\end{description}

\section{Difficultés rencontées lors des attaques} %TODO reformuler: juste "contre mesures"? mais il y a des difficultés "naturelles" et d'autres voulues
Ça c'est basique mais maintenant:

\paragraph{possibilité d'accès en direct au moment du chiffrement?} attaque synchrone/asynchrone~\cite{osvik2006cache}
\paragraph{mesure du temps:} facile si multithread. Différentes méthodes: statistiques (nécessite de bien connaître l'archi), ou bien on crée un cache hit et on mesure, puis on crée un cache miss et on mesure.
\paragraph{présence de bruit:} rend difficile la récupération des traces du chiffrement et donc ..
\paragraph{évolutions techniques qui créent des difficultés:} virtualisation~\cite{weiss2012cache}, multi coeurs. 
\paragraph{difficile pour AES sur x86}~\cite{mowery2012aes} 
\begin{itemize}
\item set d'instruction AES-NI (AES plus dans le cache mais en hard)
\item caches L1 L2
\end{itemize}
+ cf "NouvellesArchitectures.txt"
\paragraph{contre mesures avancées:}  puces crypto dédiées, algos différents.. cf papiers "contre mesures" dans biblio + Implémentation d'AES résistante aux timing attacks~\cite{kasper2009faster}

\section{Contournements}
Attaques à l'heure actuelle : mise en place, contournement des contre mesures

\section*{Conclusion}

\newpage
\nocite{*}
\bibliographystyle{plain}
\bibliography{ref}


\end{document}
