\documentclass[a4paper,11pt]{article}

\usepackage[francais]{babel}
\usepackage[utf8]{inputenc}
\usepackage[pdftex]{graphicx}
\usepackage{hyperref}
\usepackage{color}
\usepackage{fancyheadings}
\usepackage{lastpage}
\usepackage{fullpage}

\newcommand{\HRule}{\rule{\linewidth}{0.5mm}}
\newcommand{\reporttitle}{Titre}
\newcommand{\reportauthor}{Auteurs}
\newcommand{\reportdate}{\today}

\lhead{}
\rhead{\leftmark \rightmark}
\cfoot{Page \thepage/\pageref{LastPage}}


\begin{document}	

\begin{center}

\begin{minipage}[t]{0.4\textwidth}
  \begin{flushleft} \large
    \reportauthor
  \vfill
  \end{flushleft}
\end{minipage}
\begin{minipage}[t]{0.5\textwidth}
  \begin{flushright}
  \includegraphics [width=30mm]{figures/tpt.jpg} \\[0.5cm]
  \end{flushright}
\end{minipage}
\HRule \\[0.5cm]
{\huge \bfseries \reporttitle}\\[0.3cm]
\HRule \\[1.5cm]

\end{center}

\section*{Introduction}

\section{Fonctionnement du cache}
"2.1 Data Caches"~\cite{brumley2009cache}\\
"2.1. Memory and Cache Structure"~\cite{tromer2010efficient}

\section{Fonctionnement des attaques}

à chaque fois dire quelles informations on récupère

\subsection{Cas d'AES}
\begin{description}
\item[on récupère] la clé AES
\item[pourquoi] difficulté d'implémenter algo en temps constant (détailler les raisons)
\item[comment] détailler~\cite{bernstein2005cache} + (\cite{tromer2010efficient} et~\cite{osvik2006cache}(mêmes auteurs)) et contre mesures 
\end{description}

\subsection{elliptic curve portion of OpenSSL}
pertinent?~\cite{brumley2009cache}

\subsection{RSA, Diffie-Hellman}
~\cite{kocher1996timing}

\section{Contre-mesures}
A mettre avec le fonctionnement des attaques? (cf tous les papiers qui expliquent les attaques)\\
Implémentation d'AES résistante aux timing attacks~\cite{kasper2009faster}

\section*{Conclusion}

\newpage
\nocite{*}
\bibliographystyle{plain}
\bibliography{ref}


\end{document}
